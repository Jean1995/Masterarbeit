\chapter{Integration of rare processes}

\section{Muon pair production}

\begin{figure}
	\centering
	\begin{tikzpicture}
  \centering
   % Sizes
   \pgfmathsetmacro{\len}{0.05cm}
   \pgfmathsetmacro{\halflen}{\len/4}
   \pgfmathsetmacro{\vertexsize}{\len/20}
   \begin{feynman}
       % vertices
       \vertex (a) at (0, 0);
       \vertex (b) at (0, -1*\len);
       \vertex (d) at (-0.5*\len, 0.5*\len);
       \vertex (c) at (-0.5*\len, -1.5*\len);
       \vertex (i1) at (-1.5*\len, 0.5*\len);
       \vertex (i2) at (0, 1.5*\len);
       \vertex (f1) at (\len, 0.5*\len);
       \vertex (f2) at (\len, -1.5*\len);
       \vertex (f3) at (0.5, 1*\len);
       \vertex (z1) at (-1.25*\len, -1.75*\len);
       \vertex (z2) at (0.25, -1.75*\len);
 
       % draw diagram
       \diagram* {
         (i1) -- [fermion] (d) -- [fermion] (f3),
         (d) -- [boson] (a),
         (f1) -- [fermion] (a),
         (a) -- [fermion] (b),
         (b) -- [fermion] (f2),
         (b) -- [boson] (c),
       };
       \draw[thick, double] (z1) -- (c) -- (z2);
 
       % labels
       \node[left] at (i1) {$\mu^-$};
       \node[right] at (f3) {$\mu^-$};
       \node[right] at (f1) {$\mu^+$};
       \node[right] at (f2) {$\mu^-$};
       \node[left] at (z1) {$Z$};
       \node[right] at (z2) {$Z$};
  \end{feynman}
\end{tikzpicture}
    \caption{One possible feynman diagram describing the creation of muon pairs by an ingoing muon.}
    \label{fig:feynman_mupair}
\end{figure}

\begin{figure}
    \centering
    \includegraphics[scale=1]{build/dEdx_mupair.pdf}
    \caption{Continuous energy losses of muons in ice. No energy cuts are applied in this plot, hence this plot represents the case where all losses are treated continuous.  }
    \label{fig:dEdx_mupair}
\end{figure}

\begin{figure}
    \centering
    \includegraphics[scale=1]{build/spectrum_mupair.pdf}
    \caption{Secondary particle spectrum for $\num{e5}$ muons with an initial energy of $\SI{e8}{\mega\electronvolt}$, propagated in ice. The histogram shows the frequency of the stochastic losses during propagation, classified by the type of energy loss. The energy cuts applied here are $e_\text{cut} = \SI{500}{\mega\electronvolt}$, $v_\text{cut} = 0.05$.}
    \label{fig:spectrum_mupair}
\end{figure}