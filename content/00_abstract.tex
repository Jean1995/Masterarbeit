\thispagestyle{plain}

\section*{Abstract}

PROPOSAL is a \CC Monte Carlo simulation library used to describe the propagation of highly energetic particles. 
These particles can, for example, be induced by atmospheric air showers or produced from interactions of astrophysical high-energy neutrinos with matter, observed with experiments such as the IceCube Neutrino Observatory. 
In this thesis, both the implementation of photon propagation as well as a more precise description of electron and positron propagation in PROPOSAL are presented.
This allows PROPOSAL to be used as a propagator for the simulation of electromagnetic shower components, for example in the upcoming eighth version of the extensive air shower simulation software CORSIKA. 
Furthermore, rare interactions, although negligible for the average energy loss of muons but producing possibly significant detector signatures for specific analyses, are added as optional processes in PROPOSAL.

\section*{Kurzfassung}
\begin{german}

PROPOSAL ist eine \CC Monte Carlo Simulationsbibliothek zur Propagation hochenergetischer Teilchen.
Diese Teilchen entstehen beispielsweise in Luftschauern oder durch die Wechselwirkung hochenergetischer, astrophysikalischer Neutrinos mit Materie.
Experimente wie das IceCube Neutrino Observatory sind in der Lage Teilchen dieser Art zu beobachten.
In dieser Arbeit werden sowohl die Implementation der Propagation von Photonen als auch eine verbesserte Beschreibung der Propagation von Elektronen und Positronen in PROPOSAL vorgestellt.
Hierdurch ist es möglich, PROPOSAL als Propagator für die Simulation elektromagnetischer Schauer, beispielsweise in der kommenden achten Version des Luftschauer-Simulationprogrammes CORSIKA, zu verwenden.
Des Weiteren wird die Einbindung seltener Prozesse als zusätzliche, optionale Wechselwirkungen in PROPOSAL beschrieben.
Diese Prozesse sind zwar vernachlässigbar für den durchschnittlichen Energieverlust von Myonen in Materie, können gleichzeitig jedoch für einzelne Analysen relevante Detektorsignaturen verursachen. 

\end{german}
