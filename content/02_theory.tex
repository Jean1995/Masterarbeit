\chapter{Theory}

\section{The lepton propagator PROPOSAL}

PROPOSAL (\textbf{Pr}opagator with \textbf{o}ptimal \textbf{p}recision and \textbf{o}ptimized \textbf{s}peed for \textbf{a}ll \textbf{l}eptons) is a Monte Carlo simulation library capable of simulating the interactions of high energy leptons.
The original program called MMC (\textbf{M}uon \textbf{M}onte \textbf{C}arlo) has been written in the programming language Java focussing on a precise but also fast muon and tau propagation \cite{chirkin2004propagating}.
On this basis, MMC has been rewritten within a dissertation to create the \CC library PROPOSAL \cite{Kohne:2013zbq}.
To allow for a more universal use of the program, PROPOSAL can be used in Python through a wrapper as well.
More modern programming concepts such as polymorphism and a modular code structure were introduced in a recent update of PROPOSAL \cite{dunsch_2018_proposal_improvements}.

The recent version of the code is publicy available on GitHub\footnote{\url{https://github.com/tudo-astroparticlephysics/PROPOSAL}} and can be used under the terms of a modified LGPL license.
Examples of applications are the neutrino observatories IceCube and RNO who use PROPOSAL as a part of their simulation chain.