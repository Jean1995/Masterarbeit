\chapter{Theory}

\section{The lepton propagator PROPOSAL}

PROPOSAL (\textbf{Pr}opagator with \textbf{o}ptimal \textbf{p}recision and \textbf{o}ptimized \textbf{s}peed for \textbf{a}ll \textbf{l}eptons) is a Monte Carlo simulation library capable of simulating the interactions of high energy leptons.
The original program called MMC (\textbf{M}uon \textbf{M}onte \textbf{C}arlo) has been written in the programming language Java focusing on a precise but also fast muon and tau propagation \cite{chirkin2004propagating}.
On this basis, MMC has been rewritten within a dissertation to create the \CC library PROPOSAL \cite{Kohne:2013zbq}.
To allow for a more universal use of the program, PROPOSAL can be used in Python through a wrapper as well.
More modern programming concepts such as polymorphism and a modular code structure were introduced in a recent update of PROPOSAL \cite{dunsch_2018_proposal_improvements}.

The current version of the code is publicly available on GitHub\footnote{\url{https://github.com/tudo-astroparticlephysics/PROPOSAL}} and can be used under the terms of a modified LGPL license.
Examples of applications are the neutrino observatories IceCube and RNO who use PROPOSAL as a part of their simulation chain.

\subsection{Calculation of energy losses}

Energy losses of particles form the basis for the propagation algorithm in PROPOSAL.
Assuming a particle with an initial energy $E_i$, an energy loss is determined by its absolute value

\begin{equation}
	\nu = E_i \cdot v
\end{equation}

where $v$ describes the relative energy loss of the particle and $E_f = E_i - \nu$ the final particle energy.
Processes that cause energy losses and are implemented in PROPOSAL are

\begin{itemize}
	\item bremsstrahlung,
	\item ionization,
	\item photonuclear interactions and
	\item pair production of an electron-positron pair.
\end{itemize}

Quantitatively, the interaction probability for a process is described by its cross section $\sigma$.
To describe the process with respect to a specific variable in the final state, the cross section can be written in a differential form, for example $\sfrac{\symup{d}\sigma}{\symup{d}v}$.

In principle, this information could be used right away to sample energy losses from differential cross sections, which are treated as probability density functions, by using inverse sampling.
However, this approach would cause two immediate problems:
Firstly the propagation process would be very time inefficient since small energy losses, especially below the energy threshold of a detector, would be sampled individually.
Secondly numerical problems could occur due to the nature of the bremsstrahlung interaction:
Since photons are massless, the bremsstrahlung cross sections diverges for $v \to 0$, making inverse sampling over the whole parameter range impossible.

As a solution, PROPOSAL differentiates between continuous and stochastic energy losses.
The energy cut parameter $v_\text{cut}$ is defined as

\begin{equation}
	\label{eqn:cut}
	v_\text{cut} = \text{min}\left[\sfrac{e_\text{cut}}{E}, v\prime_\text{cut} \right]
\end{equation}

with a relative energy cut $v\prime_\text{cut}$ and an absolute energy cut $e_\text{cut}$.
Energy losses with $v < v_\text{cut}$ are treated as continuous losses, energy losses with $v > v_\text{cut}$ are treated as stochastic losses.
The definition in \eqref{eqn:cut} ensures that losses above an absolute detector threshold $e_\text{cut}$ are treated as stochastic even if their relative value is too small.

The propagation algorithm in PROPOSAL consists of numerous propagation steps where each step consists of continuous losses and a stochastic loss, see chapter \ref{sec:algorithm} for a detailed description.
To perform one propagation step, it is therefore necessary to have a mathematical expression to sample stochastic losses.

Let $E_i$ be the initial energy of a particle and

\begin{equation}
	\label{eqn:cum}
	P\left(E_f \leq E \leq E_i\right) = - \int_{E_i}^{E_f} p(E) \, \symup{d}E
\end{equation}

a cumulative distribution function describing the probability for a stochastic loss at a particle energy $E \geq E_f$.
With inverse sampling, this function can be used to sample the energy of the occurrence of the next stochastic loss.

To derive an expression for \eqref{eqn:cum}, the distance between the initial particle position $x_i$ and the position of the stochastic loss is discretized into sections of $\Delta x$.
The probability for a stochastic loss after a propagation distance of $x_f$, without any stochastic losses in the interval $\left(x_i, x_f\right)$, can be described as

\begin{equation}
	\label{eqn:discrete}
	\begin{split}
	\Delta P\left(x_f\right) &= P\left( x_f + \Delta x \right) - P\left( x_f \right)\\
	&= \left( 1 - \sigma(x_i) \Delta x \right) \cdot \left( 1 - \sigma(x_{i+1}) \Delta x \right) \cdot\ldots\cdot \left( 1 - \sigma(x_{f-1}) \Delta x \right) \cdot \sigma(x_f) \Delta x \\
	&\approx \exp \left( - \sum_{j=i}^{f-1} \sigma(x_j) \Delta x_j \right) \cdot \sigma(x_f) \Delta x\\
	\intertext{where $\sigma$ describes the propability for a stochastic loss. Note that $\Delta x \ll 1$ was used in the last step. In a differential form, the relation can be written as}
	\symup{d}P\left( x_f \right) &= \exp\left(-\int_{x_i}^{x_f} \sigma(x) \, \symup{d}x \right) \cdot \sigma(x_f) \, \symup{d} x_f
	\end{split}
\end{equation}

To transfer the dependency on the location $x$ to a dependency on the energy $E$, the relation

\begin{equation}
	\label{eqn:fE}
	f(E) = -\frac{\symup{d}E}{\symup{d}x} = E \frac{N_A}{A} \int_{v_\text{min}}^{v_\text{cut}} \frac{\symup{d}\sigma}{\symup{d}v} \, \symup{d}v
\end{equation}

is introduced.
Here, $f(E)$ describes the continuous energy losses between two stochastic losses and is calculated by taking the average energy loss for all interactions below the energy cut $v_\text{cut}$.
Applying \eqref{eqn:fE} on \eqref{eqn:discrete} yields

\begin{equation}
	\label{eqn:discrete_energy}
	\symup{d}P\left(E_f\right) = \exp \left( \int_{E_i}^{E_f} \frac{\sigma(E)}{ f(E) } \, \symup{d}E \right) \cdot \frac{\sigma(E_f)}{-f(E_f)} \, \symup{d}E_f.
\end{equation}

The cumulative distribution function is obtained by integrating over the probabilities in \eqref{eqn:discrete_energy}:

\begin{equation}
	\label{eqn:cum_detail}
	\begin{split}
	P\left(E_f \leq E \leq E_i\right) &= \int_{P(E_i)=0}^{P(E_f)} \symup{d}P(E_f)\\
	&= \int_{E_i}^{E_f} \exp \left( \int_{E_i}^{E_f'} \frac{\sigma(E)}{ f(E) } \, \symup{d}E \right) \cdot \frac{\sigma(E'_f)}{-f(E'_f)} \, \symup{d}E'_f.
	\end{split}
\end{equation}

The expression in \eqref{eqn:cum_detail} is simplified by using the substitution

\begin{align}
	v(E) &= \int_{E_i}^{E} \frac{\sigma(E')}{f(E')} \, \symup{d}E', & \symup{d}v &= \frac{\sigma(E)}{f(E)} \, \symup{d}E
\end{align}

where the fundamental theorem of calculus has been applied to obtain the expression for $\symup{d}v$.
It follows that

\begin{equation}
	\label{eqn:cum_final}
	\begin{split}
	P\left(E_f \leq E \leq E_i\right) &= - \int_{E_i}^{E_f} \exp \left( v(E_f') \right) \, \symup{d}v \\
	&= \left[ \exp \left( v(E'_f) \right) \right]_{E_i}^{E_f}\\
	&= \exp\left( v(E_f) \right) - \underbrace{\exp \left( v(E_i) \right)}_{= 0} \\
	&= \exp{\left( \int_{E_i}^{E_f} \frac{\sigma(E)}{f(E)} \, \symup{d}E \right)}.
	\end{split}
\end{equation}

By replacing the probability $P$ in \eqref{eqn:cum_final} by a random number $\xi \in \left(0, 1\right]$ we the energy integral

\begin{equation}
	\label{eqn:energy_integral}
	\int_{E_i}^{E_f} \frac{\sigma(E)}{-f(E)} \, \symup{d}E = - \log{\xi},
\end{equation}

originally derived in \cite{chirkin2004propagating}, is obtained. 
By sampling $\xi$, \eqref{eqn:energy_integral} can be used to calculate the energy of the occurrence of the next stochastic loss.

\subsection{Propagation algorithm}
\label{sec:algorithm}

The task of the propagation algorithm of PROPOSAL is to simulate the properties of the secondary particles produced in interactions as well as the properties of the initial particle after each interaction.
This includes information on the energy, position, direction and time of the initial particle and the secondary particles.

On the technical note, the structure of the propagation process in PROPOSAL is determined by the concept of a "chain of responsibility".
Part of this chain are the \emph{Sector} objects and a \emph{Propagator} object.

Each \emph{Sector} is defined by its geometry, its medium, its energy cut settings and other sector-specific properties.
The cut settings itself differentiate between various particle positions with respect to a predefined \emph{Detector} geometry.
By having sectors with varying characteristics the user has the possibility to appropriately model the simulation environment.

The \emph{Propagator} object chooses which sector is responsible for the propagation of the particle at its current position.
The assigned \emph{Sector} then propagates the particle within its borders and returns the propagated particle back to the \emph{Propagator}.
This process is repeated either until the propagated distance of the initial particle surpasses a preset maximal propagation distance $d_\text{max}$ or until the initial particle energy falls below a preset threshold energy $e_\text{low}$.

The following steps give a simplified overview of the propagation process within a \emph{Sector}.

\textbf{Energy of next interaction}

According to \eqref{eqn:energy_integral}, the energy of the occurrence of the next stochastic energy loss is calculated with a random number $\xi$.
If

\begin{equation}
	\xi > \exp{\left( \int_{E_i}^{e_\text{low}} \frac{\sigma(E)}{f(E)} \, \symup{d}E \right)}
\end{equation}

the sampled energy of the next stochastic loss would fall below the threshold energy $e_\text{low}$, in this case there is no stochastic loss.

Furthermore, based on the lifetime $\tau$ of the initial particle, an energy where the particle decays is sampled.
Both energy values are compared and the higher energy value, together with its interaction type (stochastic loss or decay), are used for the next step\footnote{If a decay is the next interaction, the step "Simulation of the stochastic energy loss" is replaced accordingly by a decay method.}.

\textbf{Particle displacement}

Given the initial energy $E_i$ and the energy of the interaction $E_f$, the (straight-lined) displacement is calculated with the tracking integral

\begin{equation}
	\label{eqn:tracking_integral}
	x_f = x_i - \int_{E_i}^{E_f} \frac{\symup{d}E}{f(E)}
\end{equation}

where $x_f - x_i$ describes the propagated distance.
If the calculated propagated distance would exceed the distance to the sector border $d$, $E_f$ is recalculated by setting $x_f = x_i + d$ in \eqref{eqn:tracking_integral} and solving the integral equation for $E_f$.
In this case, no interaction will occur at $E_f$.

The elapsed time it determined by the time integral 

\begin{equation}
	t_f = t_i + \int_{x_i}^{x_f} \frac{\symup{d}x}{v(x)} = t_i - \int_{E_i}^{E_f} \frac{\symup{d}E}{f(E)v(E)}
\end{equation}

with the particle velocity $v(E)$, or alternatively using the approximation $v=c$ with 

\begin{equation}
	t_f = t_i + \frac{x_f - x_i}{c}.
\end{equation}

Optionally, PROPOSAL can apply multiple scattering effects on the calculated displacement.
This changes the position of the next stochastic loss by applying a sampled deflection angle as well as the new direction of the particle.
Currently, three different parametrization for multiple scattering can be used in PROPOSAL:
A parametrization based on Molière's theory of multiple scattering as well as two parametrization based on a gaussian-like approximation of the Molière theory by Highland, see \cite{GeiselBrinck2013RevisionOT} for a detailed description of the scattering parametrization used in PROPOSAL. 

\textbf{Continuous energy losses and continuous randomization}

The energy loss between $E_i$ and $E_f$ is treated continuously according to \eqref{eqn:fE}, meaning that the particle energy is set to $E = E_f$.
However, this can cause discontinuities in the energy spectrum as shown in figure \ref{fig:cont_rand}

\begin{figure}
    \centering
    \includegraphics[scale=1]{build/cont_rand.pdf}
    \caption[]{Energy spectrum of \num{e5} muons with a starting energy of \SI{e8}{\mega\electronvolt}, propagated in \SI{300}{\meter} of standard rock\footnotemark. The spectrum show the effects of an energy cut with or without continuous randomization.}
    \label{fig:cont_rand}
\end{figure}
\footnotetext{Standard rock means a material with $Z = 11$, $A=22$ and a density of $\rho = \SI{2.65}{\gram\per\centi\metre^3}$, see e.g.\ \cite{PhysRevD.98.030001} for a detailed list of material properties.}

For a sufficiently large $v_\text{cut}$, for example $v_\text{cut} = 0.05$ in figure \ref{fig:cont_rand}, a peak in the final muon energy spectrum appears.
This peaks corresponds to all muons without any stochastic losses within the propagation distance.
These particle all have the same energy loss since no random numbers were effectively used to calculate their final energy, meaning that no fluctuations of the continuous losses are taken into account.
Setting the energy cut to a significantly lower value, for example $\v_\text{cut} = \num{e-4}$ in figure \ref{fig:cont_rand}, eliminates this peaks, however the runtime for the propagation is increased by at least an order of magnitude.

As a more time-efficient solution the option \emph{continuous randomization} can be enabled in PROPOSAL.
This applies fluctuations on the continuous loss energy sampled from a gaussian distribution.
The mean of this distribution corresponds to $E_i - E_f$, the variance is calculated by

\begin{equation}
	\left< \Delta (\Delta E)^2 \right> = \int_{E_i}^{E_f} \frac{E^2}{-f(E)} \left< \frac{\symup{d}E^2}{\symup{d}x} \right>,
\end{equation}

where the derivation of the variance follows similar steps to the derivation of \eqref{eqn:energy_integral}, see \cite{chirkin2004propagating} for a detailed derivation and description. 
The effects can be seen in figure \ref{fig:cont_rand}, the energy spectrum becomes continuous and the running time behaves similarly to the running time for the propagation without continuous randomization.

\textbf{Simulation of the stochastic energy loss}

If the stochastic energy loss falls inside the sector and occurs before the initial particle decays, a stochastic loss at the energy $E_f$ is sampled.
The total stochastic cross section for the process $i$ is calculated by

\begin{equation}
	\sigma_{\text{stoch},i}(E_f) \propto \int_{v_\text{cut}}^{v_\text{max}} \frac{\symup{d}\sigma_i(E_f)}{\symup{d}v} \, \symup{d}v.
\end{equation}

Using a random number, the occurring process is calculated where the ratios of the process probabilities are represented by the ratios of the corresponding total stochastic cross sections.
To calculate the relative size $v$ of the stochastic loss, the integral equation

\begin{equation}
	\frac{1}{\sigma_{\text{stoch},i}} \int_{v_\text{cut}}^{v} \frac{\symup{d}\sigma_i}{\symup{d}v} \, \symup{d}v = \xi
\end{equation}

is solved for $v$ where $\xi \in \left[0,1\right)$ is a random number and $i$ the selected process.

The propagation routine is repeated with the sampling of the energy of the next interaction until the particle has decayed, has reached the sector border or until its energy has reached the threshold energy $e_\text{low}$.

\subsection{Muon propagation with PROPOSAL}

%%% Macros to include numbers in tables
\CatchFileDef{\epaircount}{build/numbers/epair_count.tex}{}
\CatchFileDef{\bremscount}{build/numbers/brems_count.tex}{}
\CatchFileDef{\photocount}{build/numbers/photo_count.tex}{}
\CatchFileDef{\ionizcount}{build/numbers/ioniz_count.tex}{}

\CatchFileDef{\epairsum}{build/numbers/epair_esum.tex}{}
\CatchFileDef{\bremssum}{build/numbers/brems_esum.tex}{}
\CatchFileDef{\photosum}{build/numbers/photo_esum.tex}{}
\CatchFileDef{\ionizsum}{build/numbers/ioniz_esum.tex}{}

At the end of the propagation process, PROPOSAL returns the properties of the produced secondary particles as well as the final properties of the initial particle or, if it did decay during propagation, its decay products.
In this chapter the characteristic energy losses of muons are described, where ice is used exemplarily as a medium for all plots. The parametrizations for the interactions are always the default options in PROPOSAL, the Landau-Pomeranchuk-Migdal (LPM) effect for bremsstrahlung and pair production has been enabled. \todo{Source for LPM effect}

In figure \ref{fig:dEdx} the continuous energy losses of muons in ice, calculated according to \eqref{eqn:fE}, are shown.
For this plot the energy cut has been set to $v_\text{cut} = v_\text{max}$, therefore the values shown correspond to the complete average energy losses of muons in ice. 
It can be seen that the average energy loss is quantitatively dominated by ionization for lower energies while §e§ pair production, bremsstrahlung and photonuclear interactions become dominant for higher energies.
Furthermore, it can clearly be seen that the parametrization

\begin{equation}
	- \left\langle \frac{\symup{d}E}{\symup{d}x} \right\rangle \approx a(E) + b(E) \cdot E
\end{equation}

of the average energy loss as a quasi-linear function is valid where $a(E)$ corresponds to energy losses due to ionization and $b(E)$ to energy losses due to $e$ pair production, bremsstrahlung and photonuclear interactions.
The parameters $a(E), b(E)$ vary only slightly with energy.

\begin{figure}
    \centering
    \includegraphics[scale=1]{build/dEdx.pdf}
    \caption{Continuous energy losses of muons in ice. No energy cuts are applied in this plot, hence this plot represents the case where all losses are treated continuous.  }
    \label{fig:dEdx}
\end{figure}

In figure \ref{fig:spectrum} and \ref{fig:secondary_number} the stochastic losses for muons propagated in ice are shown, the energy cuts used here are $e_\text{cut} = \SI{500}{\mega\electronvolt}$ and $v_\text{cut} = \num{0.05}$ which corresponds to the energy cuts used in the analysis of the IceCube experiment.
In both figures, the muons are propagated until they decay.

The histogram in figure \ref{fig:spectrum} shows the energies of all secondary particles sorted by interaction type for $\num{e5}$ muons propagated with an initial energy of $\SI{e8}{\mega\electronvolt}$.
Between about $\SI{e3}{\mega\electronvolt}$ and $\SI{e6}{\mega\electronvolt}$ the energy losses are dominated by $e$ pair production, this dominance could for example be used to probe the pair production parametrization in this energy range.
For higher secondary energies, bremsstrahlung and photonuclear interactions are the dominant effects.
Another effect that can be seen it the energy cut at $e_\text{cut} = \SI{500}{\mega\electronvolt}$ where the histogram cuts off abruptly.
The energy losses below $e_\text{cut}$ correspond to losses where $E \cdot v < e_\text{cut}$ but $v > v_\text{cut} = 0.05$.
These losses are dominated by ionization losses since ionization is the relevant process for low energies.

\begin{figure}
    \centering
    \includegraphics[scale=1]{build/spectrum.pdf}
    \caption{Secondary particle spectrum for $\num{e5}$ muons with an initial energy of $\SI{e8}{\mega\electronvolt}$, propagated in ice. The histogram shows the frequency of the stochastic losses during propagation, classified by the type of energy loss. The energy cuts applied here are $e_\text{cut} = \SI{500}{\mega\electronvolt}$, $v_\text{cut} = 0.05$.}
    \label{fig:spectrum}
\end{figure}

The two-dimensional histograms in figure \ref{fig:secondary_number} show the sorted energy losses correlated with the energy of the initial particle at the time the interaction.
Here, $\num{e3}$ muons with an initial energy of $\SI{e14}{\mega\electronvolt}$ are propagated.
Table \ref{tab:secondary_number} additionally shows the sum of the secondary energy losses as well as the frequency of the energy losses for every possible interaction.

It can be seen that bremsstrahlung and photonuclear interaction tend to have a more homogeneous spectrum where the secondary energy is less uncorrelated with the primary energy than for ionization and pair production.
For bremsstrahlung, the effects of the LPM effect can be seen since this effect causes the bremsstrahlung cross section to be suppressed for small $v$ at very high energies.
Especially for the ionization histogram the effects of the combined $e_\text{cut}$ and $v_\text{cut}$ can be seen for small primary energies leading to secondary energies below $e_\text{cut}$.
Table \ref{tab:secondary_number} shows that the sum of the energy losses are of the same order of magnitude for pair production, bremsstrahlung and photonuclear interaction while the contribution for ionization is significantly lower since the latter is mainly treated continuously.
Although the energy loss contribution of pair production is comparable to bremsstrahlung and photonuclear interaction its frequency is of several orders of magnitude higher due to its tendency to produce energy losses with a smaller relative energy loss.

\begin{figure}
    \centering
    \includegraphics[scale=1]{build/secondary_number.pdf}
    \caption{Energy spectra for $\num{e3}$ muons with an initial energy of $E = \SI{e14}{\mega\electronvolt}$ propagated in ice. For each histogram, the x-axis shows the energy of the primary particle during the stochastic loss and the y-axis the energy of the secondary particle created in the stochastic loss. The energy cuts applied here are $e_\text{cut} = \SI{500}{\mega\electronvolt}$, $v_\text{cut} = 0.05$.}
    \label{fig:secondary_number}
\end{figure}
\todo{Adapt numbers in captions of figures}

\begin{table}
	\centering
	\caption[]{Interaction-specific frequency and sum of stochastic energy losses according to figure \ref{fig:secondary_number}.}
	\label{tab:secondary_number}
	\sisetup{
  		output-exponent-marker = \text{e},
  		table-format=+1.2e+2,
  		exponent-product={},
  		retain-explicit-plus
	}	
	\begin{tabular}{l S S}
		\toprule
		{Interaction} & {Frequency} & {$\sum E_\text{prim} \cdot v \:/\: \si{\mega\electronvolt}$} \\	
		\midrule
		\text{pair production} & \epaircount & \epairsum \\
		\text{Bremsstrahlung} & \bremscount & \bremssum \\
		\text{Photonuclear} & \photocount & \photosum \\
		\text{Ionization} & \ionizcount & \ionizsum \\
		\bottomrule
	\end{tabular}
\end{table}


