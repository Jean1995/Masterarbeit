\chapter{Large scale Monte Carlo simulation software}

\label{sec:theory}

Modern physical experiments often produce extremely large quantities of data.
In order to obtain physical results, these datasets need to be evaluated where using computational methods of multivariate analyses is often inevitable due to the complexity and multiplicity of the data.

One important task that needs to be performed is the signal-background separation to reject events that are irrelevant for the conducted analysis.
Examples are the gamma-hadron separation in high-energy gamma ray astronomy \cite{Ohm_2009} or the separation of atmospheric muons from neutrino-induced muons in the IceCube Neutrino Observatory \cite{Ahrens_2004}. \todo{Are there better papers to cite here?}
Signatures from signal and background events are separated by classifiers, deciding whether an event is labeled as signal or background, for example by random forests.
To be able to use a classifier on a given dataset, the models needs to be trained first using data from Monte Carlo simulations where it is known whether the generated signature belongs to a signal or a background event.
When generating Monte Carlo simulations, both a high quality and quantity of simulation data are important and need to be taken into account when writing a simulation program.

Two examples of Monte Carlo simulation softwares, the lepton propagator PROPOSAL and the air shower simulation program CORSIKA, are presented in the following section.
A common characteristic of both programs is that particles traversing large distances need to be simulated both correctly and efficiently, requiring a trade-off between accuracy and computation time.

\section{The lepton propagator PROPOSAL}

PROPOSAL (\textbf{Pr}opagator with \textbf{O}ptimal \textbf{P}recision and \textbf{O}ptimized \textbf{S}peed for \textbf{A}ll \textbf{L}eptons) is a Monte Carlo simulation library capable of simulating the interactions of high energy leptons.
The original program called MMC (\textbf{M}uon \textbf{M}onte \textbf{C}arlo) has been written in the programming language Java focusing on a precise but also fast muon and tau propagation \cite{chirkin2004propagating}.
On this basis, MMC has been rewritten within a dissertation to create the \CC library PROPOSAL \cite{Kohne:2013zbq}.
To allow for a more universal use of the program, PROPOSAL can be used in Python through a wrapper as well.
More modern programming concepts such as polymorphism and a modular code structure were introduced in a recent update of PROPOSAL \cite{dunsch_2018_proposal_improvements}.

The current version of the code is publicly available on GitHub\footnote{\url{https://github.com/tudo-astroparticlephysics/PROPOSAL}} and can be used under the terms of a modified LGPL license.
Examples of applications are the neutrino observatories IceCube and RNO who use PROPOSAL as a part of their simulation chain \cite{Khokhlov:2019pgh}.

\subsection{Calculation of energy losses}
\label{sec:energy_loss_calculation}

Energy losses of particles form the basis for the propagation algorithm in PROPOSAL.
Assuming a particle with an initial energy $E_i$, an energy loss is described by its absolute value
\begin{equation}
	\nu = E_i \cdot v
\end{equation}
%
where $v$ describes the relative energy loss of the particle and $E_f = E_i - \nu$ the final particle energy.
Processes causing energy losses and implemented are
%
\begin{itemize}
	\item bremsstrahlung \cite{brems_default},
	\item ionization \cite{Rossi:99081, PhysRevD.98.030001},
	\item photonuclear interactions \cite{abramowicz1997allm, butkevich2001crosssection} and
	\item pair production of an electron-positron pair \cite{epair_correction, osti_4563918},
\end{itemize}
%
with references to descriptions of the default parametrizations used in PROPOSAL.
Alternative parametrizations, for example recently improved bremsstrahlung and pair production cross sections provided by \cite{Soedingrekso:2019qlr}, are available as well.

Quantitatively, the interaction probability for a process is described by its cross section $\sigma$.
To describe the process with respect to a specific variable in the final state, the cross section can be written in a differential form, for example $\sfrac{\symup{d}\sigma}{\symup{d}v}$.

In principle, this information could be used right away to sample energy losses from differential cross sections, which are treated as probability density functions, by using inverse sampling.
However, this approach would cause two immediate problems:
Firstly, the propagation process would be very time inefficient since small energy losses, especially below the energy threshold of a detector, would be sampled individually.
Secondly, numerical problems will occur due to the nature of the bremsstrahlung interaction:
Since photons are massless, the bremsstrahlung cross sections diverges for $v \to 0$, making inverse sampling over the whole parameter range impossible.

As a solution, PROPOSAL differentiates between continuous and stochastic energy losses.
The energy cut parameter $v\prime_\text{cut}$ is defined as
%
\begin{equation}
	\label{eqn:cut}
	v\prime_\text{cut} = \text{min}\left[\sfrac{e_\text{cut}}{E}, v_\text{cut} \right]
\end{equation}
%
with a relative energy cut $v_\text{cut}$ and an absolute energy cut $e_\text{cut}$.
Energy losses with $v > v\prime_\text{cut}$ are treated as stochastic losses, meaning that every interaction with an relative energy loss above the cut is treated individually.
Energy losses with $v < v\prime_\text{cut}$ however are treated as continuous losses, meaning that an averaged energy loss per distance is calculated from all energy losses below the cut and applied to the particles during propagation.
Both parameters $e_{\text{cut}}$ and $v_{\text{cut}}$ can be set, or disabled, separately.
By enabling both parameters simultaneously, the definition in \eqref{eqn:cut} ensures that losses above an absolute detector threshold $e_\text{cut}$ are treated as stochastic even if their relative value is below $v_{\text{cut}}$.

The propagation algorithm in PROPOSAL consists of several, consecutively executed propagation steps where each step consists of continuous losses and a stochastic loss, see section \ref{sec:algorithm} for a detailed description.
To perform one propagation step, it is therefore necessary to have a mathematical expression to sample the next stochastic loss.

Let $E_i$ be the initial energy of a particle and
%
\begin{equation}
	\label{eqn:cum}
	P\left(E_f \leq E \leq E_i\right) = - \int_{E_i}^{E_f} p(E) \, \symup{d}E
\end{equation}
%
a cumulative distribution function describing the probability for a stochastic loss at a particle energy $E \geq E_f$.
With inverse sampling, this function can be used to sample the remaining particle energy $E_f$ just before the next stochastic interaction will occur.

To derive an expression for \eqref{eqn:cum}, the distance between the initial particle position $x_i$ and the position of the stochastic loss $x_f$ is discretized into sections of $\Delta x$.
The probability for a stochastic loss after a distance of $x_f - x_i$, without any stochastic losses in the interval $\left(x_i, x_f\right)$, can be described as
%
\begin{equation}
	\label{eqn:discrete}
	\begin{split}
	\Delta P\left(x_f\right) &= P\left( x_f + \Delta x \right) - P\left( x_f \right)\\
	&= \left( 1 - \sigma(x_i) \Delta x \right) \cdot \left( 1 - \sigma(x_{i+1}) \Delta x \right) \cdot\ldots\cdot \left( 1 - \sigma(x_{f-1}) \Delta x \right) \cdot \sigma(x_f) \Delta x \\
	&\approx \exp \left( - \sum_{j=i}^{f-1} \sigma(x_j) \Delta x_j \right) \cdot \sigma(x_f) \Delta x\\
	\intertext{where $\sigma$ describes the probability for a stochastic loss. Note that $\Delta x \ll 1$ was used in the last step. In a differential form, the relation can be written as}
	\symup{d}P\left( x_f \right) &= \exp\left(-\int_{x_i}^{x_f} \sigma(x) \, \symup{d}x \right) \cdot \sigma(x_f) \, \symup{d} x_f
	\end{split}
\end{equation}
%
To transfer the dependency on the location $x$ to a dependency on the energy $E$, the relation
%
\begin{equation}
	\label{eqn:fE}
	f(E) = -\frac{\symup{d}E}{\symup{d}x} = E \frac{N_A}{A} \int_{v_\text{min}}^{v\prime_\text{cut}} v \frac{\symup{d}\sigma}{\symup{d}v} \, \symup{d}v,
\end{equation}
%
with the Avogadro constant $N_A$ and the mass number $A$ of the current medium\footnote{For a medium composed of different atoms, an averaged sum for $A$ is used.}, is introduced.
Here, $f(E)$ describes the continuous energy losses between two stochastic losses and is calculated by taking the average energy loss for all interactions below the energy cut $v\prime_\text{cut}$.

Applying \eqref{eqn:fE} on \eqref{eqn:discrete} yields
%
\begin{equation}
	\label{eqn:discrete_energy}
	\symup{d}P\left(E_f\right) = \exp \left( \int_{E_i}^{E_f} \frac{\sigma(E)}{ f(E) } \, \symup{d}E \right) \cdot \frac{\sigma(E_f)}{-f(E_f)} \, \symup{d}E_f.
\end{equation}

The cumulative distribution function is obtained by integrating over the probabilities in \eqref{eqn:discrete_energy}:
%
\begin{equation}
	\label{eqn:cum_detail}
	\begin{split}
	P\left(E_f \leq E \leq E_i\right) &= \int_{P(E_i)=0}^{P(E_f)} \symup{d}P(E_f)\\
	&= \int_{E_i}^{E_f} \exp \left( \int_{E_i}^{E_f'} \frac{\sigma(E)}{ f(E) } \, \symup{d}E \right) \cdot \frac{\sigma(E'_f)}{-f(E'_f)} \, \symup{d}E'_f.
	\end{split}
\end{equation}

The expression in \eqref{eqn:cum_detail} is simplified by using the substitution
%
\begin{align}
	u(E) &= \int_{E_i}^{E} \frac{\sigma(E')}{f(E')} \, \symup{d}E', & \symup{d}u &= \frac{\sigma(E)}{f(E)} \, \symup{d}E
\end{align}
%
where the fundamental theorem of calculus has been applied to obtain the expression for $\symup{d}u$.

It follows that
%
\begin{equation}
	\label{eqn:cum_final}
	\begin{split}
	P\left(E_f \leq E \leq E_i\right) &= - \int_{E_i}^{E_f} \exp \left( u(E_f') \right) \, \symup{d}u \\
	&= \left[ \exp \left( u(E'_f) \right) \right]_{E_i}^{E_f}\\
	&= \exp\left( u(E_f) \right) - \underbrace{\exp \left( u(E_i) \right)}_{= 0} \\
	&= \exp{\left( \int_{E_i}^{E_f} \frac{\sigma(E)}{f(E)} \, \symup{d}E \right)}.
	\end{split}
\end{equation}

By replacing the probability $P$ in \eqref{eqn:cum_final} by a random number $\xi \in \left(0, 1\right]$ the energy integral
%
\begin{equation}
	\label{eqn:energy_integral}
	\int_{E_i}^{E_f} \frac{\sigma(E)}{-f(E)} \, \symup{d}E = - \log{\xi},
\end{equation}
%
originally derived in \cite{chirkin2004propagating}, is obtained. 
By sampling $\xi$, \eqref{eqn:energy_integral} can be used to calculate the remaining particle energy $E_f$ just before the next stochastic energy loss will occur.

\subsection{Propagation algorithm}
\label{sec:algorithm}

The task of the propagation algorithm of PROPOSAL is to simulate the properties of the secondary particles produced in interactions as well as the properties of the initial particle after each interaction.
This includes information on the energy, position, direction and time of the propagated particle and the secondary particles.

On the technical note, the structure of the propagation process in PROPOSAL is determined by the concept of a "chain of responsibility".
Part of this chain are the \emph{Sector} objects and a \emph{Propagator} object.

Each \emph{Sector} is defined by its geometry (i.e.\ the spatial extent of the \emph{Sector}), its medium, its energy cut settings and other sector-specific properties.
The cut settings itself differentiate between various particle positions with respect to a predefined \emph{Detector} which is a region with an increased propagation accuracy.
By having sectors with varying characteristics the user has the possibility to appropriately model the simulation environment.

The \emph{Propagator} object chooses which \emph{Sector} is responsible for the propagation of the particle at its current position.
The assigned \emph{Sector} then propagates the particle within its borders and returns the propagated particle back to the \emph{Propagator}.
This process is repeated either until the propagated distance of the initial particle surpasses a preset maximal propagation distance $d_\text{max}$ or until the initial particle energy falls below a preset threshold energy $e_\text{low}$.

The following steps give a simplified overview of the propagation process within a \emph{Sector}.

\subsubsection{Energy of the occurrence of the next interaction}

According to \eqref{eqn:energy_integral}, the remaining particle energy $E_f$ just before the next stochastic loss will occur is sampled using a random number $\xi$.
If
%
\begin{equation}
	\xi > \exp{\left( \int_{E_i}^{e_\text{low}} \frac{\sigma(E)}{f(E)} \, \symup{d}E \right)},
\end{equation}
%
the sampled energy where the next stochastic loss occurs would fall below the threshold energy $e_\text{low}$, in this case there is no stochastic loss.

If the propagated particle is able to decay, an energy where the particle decays is sampled based on its lifetime $\tau$
Both energy values are compared and the higher energy value, together with its interaction type (stochastic loss or decay), are used for the next step\footnote{If a decay is the next interaction, the step "Simulation of the stochastic energy loss" is replaced accordingly by a decay method.}.

\subsubsection{Particle displacement}

Given the initial energy $E_i$ and the energy of the interaction $E_f$, the (straight-lined) displacement is calculated with the tracking integral
%
\begin{equation}
	\label{eqn:tracking_integral}
	x_f = x_i - \int_{E_i}^{E_f} \frac{\symup{d}E}{f(E)}
\end{equation}
%
where $x_f - x_i$ describes the propagated distance.
If the calculated propagated distance would exceed the distance to the sector border $d$, $E_f$ is recalculated by setting $x_f = x_i + d$ in \eqref{eqn:tracking_integral} and solving the integral equation for $E_f$.
In this case, no interaction will occur at $E_f$.

The elapsed time is determined by the time integral 
%
\begin{equation}
	t_f = t_i + \int_{x_i}^{x_f} \frac{\symup{d}x}{v(x)} = t_i - \int_{E_i}^{E_f} \frac{\symup{d}E}{f(E)v(E)}
\end{equation}
%
with the particle velocity $v(E)$, or alternatively using the approximation $v=c$ with 
%
\begin{equation}
	t_f = t_i + \frac{x_f - x_i}{c}.
\end{equation}

Optionally, PROPOSAL can apply multiple scattering effects on the calculated displacement.
This changes the position of the next stochastic loss by sampling a deflection angle as well as the new direction of the particle.
Currently, three different parametrizations for multiple scattering can be used in PROPOSAL:
A parametrization based on Molière's theory of multiple scattering as well as two parametrizations based on a gaussian-like approximation of the Molière theory by Highland, see \cite{GeiselBrinck2013RevisionOT} for a detailed description of the scattering processes used in PROPOSAL. 

\subsubsection{Continuous energy losses and continuous randomization}

The energy loss between $E_i$ and $E_f$ is treated continuously according to \eqref{eqn:fE}, meaning that the particle energy is set to $E = E_f$.
However, this can cause discontinuities in the energy spectrum as shown in figure \ref{fig:cont_rand}.

\begin{figure}
    \centering
    \includegraphics[scale=1]{build/cont_rand.pdf}
    \caption[Energy spectrum of \num{e5} muons with a starting energy of \SI{e8}{\mega\electronvolt}, propagated in \SI{300}{\meter} of standard rock. The spectrum shows the effects of an energy cut with or without continuous randomization.]{Energy spectrum of \num{e5} muons with a starting energy of \SI{e8}{\mega\electronvolt}, propagated in \SI{300}{\meter} of standard rock\protect\footnotemark. The spectrum shows the effects of an energy cut with or without continuous randomization.}
    \label{fig:cont_rand}
\end{figure}
\todo{Check if the standard rock footnote is on the right page}
For a sufficiently large $v_\text{cut}$, for example $v_\text{cut} = 0.05$ in figure \ref{fig:cont_rand}, a peak in the final muon energy spectrum appears.
This peak corresponds to all muons without any stochastic losses within the propagation distance.
These particle all have the same final energy since no random numbers were effectively used to calculate their final energy, meaning that no fluctuations of the continuous losses are taken into account.
Setting the energy cut to a significantly lower value, for example $v_\text{cut} = \num{e-4}$ in figure \ref{fig:cont_rand}, eliminates the peak, however the runtime for the propagation is increased by at least an order of magnitude.

As a more time-efficient solution the option \emph{continuous randomization} can be enabled in PROPOSAL.
This applies fluctuations on the continuous loss energies sampled from a gaussian distribution.
The mean of this distribution corresponds to $E_i - E_f$, the variance is calculated by
%
\begin{equation}
	\left< \Delta (\Delta E)^2 \right> = \int_{E_i}^{E_f} \frac{E^2}{-f(E)} \left< \frac{\symup{d}^2E}{\symup{d}x^2} \right>,
\end{equation}
%
where the derivation of the variance follows similar steps to the derivation of \eqref{eqn:energy_integral}, see \cite{chirkin2004propagating} for a detailed derivation and description. 
The effects can be seen in figure \ref{fig:cont_rand}, the energy spectrum becomes continuous and the running time behaves similarly to the running time for the propagation without continuous randomization.

\footnotetext{Standard rock means a material with $Z = 11$, $A=22$ and a density of $\rho = \SI{2.65}{\gram\per\centi\metre^3}$, see e.g.\ \cite{PhysRevD.98.030001} for a detailed list of material properties.}

\subsubsection{Simulation of the stochastic energy loss}

If the stochastic energy loss falls inside the sector and occurs before the initial particle decays, a stochastic loss at the energy $E_f$ is sampled.
The total stochastic cross section for the process $i$ is calculated by
%
\begin{equation}
	\label{eqn:stoch_crosssection}
	\sigma_{\text{stoch},i}(E_f) \propto \int_{v\prime_\text{cut}}^{v_\text{max}} \frac{\symup{d}\sigma_i(E_f)}{\symup{d}v} \, \symup{d}v.
\end{equation}

Using a random number $\xi_1$, the occurring process is calculated where the ratios of the process probabilities are represented by the ratios of the corresponding total stochastic cross sections.
To calculate the relative size $v$ of the stochastic loss, the integral equation
%
\begin{equation}
	\label{eqn:sample_v}
	\frac{1}{\sigma_{\text{stoch},i}} \int_{v\prime_\text{cut}}^{v} \frac{\symup{d}\sigma_i}{\symup{d}v} \, \symup{d}v = \xi_2
\end{equation}
%
is solved for $v$ where $\xi_2 \in \left[0,1\right)$ is a another random number and $i$ the selected process.

The propagation routine is repeated by sampling the remaining particle energy before the next interaction occurs (i.e.\ the first step described here) until the particle has decayed, has reached the sector border or until its energy has reached the threshold energy $e_\text{low}$.

\subsection{Muon propagation with PROPOSAL}

%%% Macros to include numbers in tables
\CatchFileDef{\epaircount}{build/numbers/epair_count.tex}{}
\CatchFileDef{\bremscount}{build/numbers/brems_count.tex}{}
\CatchFileDef{\photocount}{build/numbers/photo_count.tex}{}
\CatchFileDef{\ionizcount}{build/numbers/ioniz_count.tex}{}

\CatchFileDef{\epairsum}{build/numbers/epair_esum.tex}{}
\CatchFileDef{\bremssum}{build/numbers/brems_esum.tex}{}
\CatchFileDef{\photosum}{build/numbers/photo_esum.tex}{}
\CatchFileDef{\ionizsum}{build/numbers/ioniz_esum.tex}{}

At the end of the propagation process, PROPOSAL returns the properties of the produced secondary particles as well as the final properties of the initial particle or, if it did decay during propagation, its decay products.
In this section the characteristic energy losses of muons are described, where ice is used exemplarily as a medium for all plots. The parametrizations for the interactions are always the default options in PROPOSAL, see the listing in section \ref{sec:energy_loss_calculation}, furthermore the Landau-Pomeranchuk-Migdal (LPM) effect for bremsstrahlung and pair production has been enabled, see \cite{Kohne:2013zbq} for a detailed description of the LPM effect.

In figure \ref{fig:dEdx} the continuous energy losses of muons in ice, calculated according to \eqref{eqn:fE}, are shown.
For this plot the energy cut has been set to $v_\text{cut} = v_\text{max}$, therefore the values shown correspond to the complete average energy losses of muons in ice. 
It can be seen that the average energy loss is quantitatively dominated by ionization for lower energies while $e$ pair production, bremsstrahlung and photonuclear interactions become dominant for higher energies.
Furthermore, it can clearly be seen that the parametrization
%
\begin{equation}
	- \left\langle \frac{\symup{d}E}{\symup{d}x} \right\rangle \approx a(E) + b(E) \cdot E
\end{equation}
%
of the average energy loss as a quasi-linear function is valid where $a(E)$ corresponds to energy losses due to ionization and $b(E)$ to energy losses due to $e$ pair production, bremsstrahlung and photonuclear interactions.
The parameters $a(E), b(E)$ vary only logarithmically with energy.

\begin{figure}
    \centering
    \includegraphics[scale=1]{build/dEdx.pdf}
    \caption{Continuous energy losses of muons in ice. No energy cuts are applied in this plot, hence this plot represents the case where all losses are treated continuously. Additionally, the average energy loss due to muon decay is shown where the muon loses, per definition, all its energy when it decay. }
    \label{fig:dEdx}
\end{figure}

In figure \ref{fig:spectrum} and \ref{fig:secondary_number} the stochastic losses for muons propagated in ice are shown, the energy cuts applied here are $e_\text{cut} = \SI{500}{\mega\electronvolt}$ and $v_\text{cut} = \num{0.05}$ while the muons are propagated until they decay or lost all their energy.

The histogram in figure \ref{fig:spectrum} shows the energies of all secondary particles sorted by interaction type for $\num{e5}$ muons propagated with an initial energy of $\SI{e8}{\mega\electronvolt}$.
Between about $\SI{e3}{\mega\electronvolt}$ and $\SI{e6}{\mega\electronvolt}$ the energy losses are dominated by $e$ pair production, this dominance could for example be used to probe the $e$ pair production cross section in this energy range.
For higher secondary energies, bremsstrahlung and photonuclear interactions are the dominant effects.
Another effect that can be seen it the energy cut at $e_\text{cut} = \SI{500}{\mega\electronvolt}$ where the histogram cuts off abruptly.
The energy losses below $e_\text{cut}$ correspond to losses where $E \cdot v < e_\text{cut}$ but $v > v_\text{cut} = 0.05$.
These losses are dominated by ionization losses since ionization is the relevant process for low energies.

\begin{figure}
    \centering
    \includegraphics[scale=1]{build/spectrum.pdf}
    \caption{Secondary particle energy spectrum for $\num{e5}$ muons with an initial energy of $\SI{e8}{\mega\electronvolt}$, propagated in ice. The histogram shows the frequency of the stochastic losses during propagation, classified by the type of energy loss. The energy cuts applied here are $e_\text{cut} = \SI{500}{\mega\electronvolt}$, $v_\text{cut} = 0.05$.}
    \label{fig:spectrum}
\end{figure}

The two-dimensional histograms in figure \ref{fig:secondary_number} show the sorted energy losses correlated with the energy of the initial particle at the time of the interaction.
Here, $\num{e3}$ muons with an initial energy of $\SI{e14}{\mega\electronvolt}$ are propagated.
Table \ref{tab:secondary_number} additionally shows the sum of the secondary energy losses as well as the frequency of the energy losses for every possible interaction.

It can be seen that bremsstrahlung and photonuclear interaction tend to have a more homogeneous spectrum where the secondary energy is less correlated with the primary energy than for ionization and pair production.
For bremsstrahlung, the effects of the LPM effect can be seen since this effect causes the bremsstrahlung cross section to be suppressed for small $v$ at very high energies.
Especially for the ionization histogram the effects of the combined $e_\text{cut}$ and $v_\text{cut}$ can be seen for small primary energies leading to secondary energies below $e_\text{cut}$.
Table \ref{tab:secondary_number} shows that the sum of the energy losses are of the same order of magnitude for pair production, bremsstrahlung and photonuclear interaction while the contribution for ionization is significantly lower since the latter is mainly treated continuously.
Although the energy loss contribution of pair production is comparable to bremsstrahlung and photonuclear interaction its frequency is of several orders of magnitude higher due to its tendency to produce energy losses with a smaller relative energy.

\begin{figure}
    \centering
    \includegraphics[scale=1]{build/secondary_number.pdf}
    \caption{Energy spectra for $\num{e3}$ muons with an initial energy of $E = \SI{e14}{\mega\electronvolt}$ propagated in ice. For each histogram, the x-axis shows the energy of the primary particle before the stochastic loss and the y-axis the energy of the secondary particle created in the stochastic loss. The energy cuts applied here are $e_\text{cut} = \SI{500}{\mega\electronvolt}$, $v_\text{cut} = 0.05$.}
    \label{fig:secondary_number}
\end{figure}
\todo{Adapt numbers in captions of figures}

\begin{table}
	\centering
	\caption[]{Interaction-specific frequency and sum of stochastic energy losses according to figure \ref{fig:secondary_number}.}
	\label{tab:secondary_number}
	\sisetup{
  		output-exponent-marker = \text{e},
  		table-format=+1.2e+2,
  		exponent-product={},
  		retain-explicit-plus
	}	
	\begin{tabular}{l S S}
		\toprule
		{Interaction} & {Frequency} & {$\sum E_\text{prim} \cdot v \:/\: \si{\mega\electronvolt}$} \\	
		\midrule
		\text{pair production} & \epaircount & \epairsum \\
		\text{Bremsstrahlung} & \bremscount & \bremssum \\
		\text{Photonuclear} & \photocount & \photosum \\
		\text{Ionization} & \ionizcount & \ionizsum \\
		\bottomrule
	\end{tabular}
\end{table}


\section{The shower simulation program CORSIKA}

\label{sec:corsika}

CORSIKA (\textbf{CO}smic \textbf{R}ay \textbf{SI}mulations for \textbf{KA}scade) is a software to simulate extensive air showers induced by high-energy particles in the atmosphere.
The software was originally developed to perform simulations for the extensive air shower experiment KASCADE with the first version of CORSIKA being released in October 1989 \cite{userguide}.
Today, CORSIKA is used as a general purpose tool for all experiments in need of shower simulation data.
At this time (February 2020), the most recent version is CORSIKA 7.7100 which has been released in October 2019.

\subsection{Description of CORSIKA 7}

CORSIKA 7 consists of routines written in the programming language FORTRAN77.
It simulates interactions of electrons, positrons, photons, muons, hadrons and nuclei necessary for shower simulations with energies of up to $\SI{e20}{\electronvolt}$.

As described in the CORSIKA user guide \cite{userguide}, the basic software structure is based on four main parts.
The first three parts are responsible for the description of particle-specific physical interactions:
Firstly, the transport of electrons, positrons and photons, i.e.\ the electromagnetic shower components, can either be described by an adapted version of the EGS4 package, a Monte Carlo software described in more detail in section \ref{sec:egs4}, or analytically using NKG formulas.
The description of the hadronic interactions of the shower is divided into a low-energy and a high-energy part.
The former can either be described by FLUKA, by GHEISHA or by UrQMD while the latter can be described by several models including QGSJET, SIBYLL or EPOS LHC.
Links to further descriptions of the individual models are provided by the CORSIKA user guide \cite{userguide}.
The last part of the basic structure of CORSIKA is a general program frame that is responsible for general tasks including particle tracking, particle decays or organizing dynamic particle data in a stack data structure.

The general process of shower propagation in CORSIKA can be described as follows:
At first, CORSIKA samples an initial particle, inducing the air shower, from a pre-defined distribution.
This particle is propagated where the treatment of the interactions depends on the interaction models chosen by the user.
Further effects during the propagation step, for example deflection by a magnetic field, are applied here as well.
Created secondary particles are saved temporarily on a stack.
After the current particle interacted, the next particle is read from the stack and propagated as well until its energy falls below a certain threshold and the particle is removed from the simulation.
Output data with information on all secondary particles on a pre-selected observation level, including particle types, locations, directions and arrival times are generated continuously during the simulation process and can afterwards be further processed by the user.
As soon as the particle stack is empty, the simulation continues by sampling another primary particle initiating the next air shower.

Additional modules changing the program sequence or adding new physical properties can be loaded into CORSIKA at compilation time.
For example, these extensions can add the simulation of Cherenkov radiation or neutrinos, enable parallel computation of the shower simulation or adapt the simulation output to conform with the requirements of specific experiments.  

\subsection{Development of CORSIKA 8}

Although originally developed only to provide simulations for the KASCADE experiment, CORSIKA evolved into a widely used program for air shower simulations.
While updates for the interaction models are regularly released, new extensions for CORSIKA are rarely developed.
One reason for the lack of new features is the complicated code structure that has originally been written without considering the possibility of new extensions.
Instead, these extensions need to be developed to conform specifically with the existing code structure.
Furthermore, CORSIKA is written in FORTRAN77 which, on the one hand, poses technical restrictions when developing new code, and on the other hand lacks a large user group compared to modern programming languages such as \CC or Python.

To overcome these existing obstacles, a new version of CORSIKA is being developed.
With a code structure written from scratch in the programming language \CC, CORSIKA 8 intends to provide a new shower simulation software with focus on flexibility and extensibility oriented towards the present use case of CORSIKA.

A description of the current status of the development of CORSIKA 8 is given by \cite{corsika8}.
Furthermore, four building blocks, describing the very general code structure, are presented.
These blocks are the particle stack, storing and giving access to all particle properties, the process sequence, describing all physical interactions, the transport code, responsible for the general propagation process as well as the environment that can be composed modularly by the user.
One main feature of the CORSIKA 8 code is to provide the possibility to easily alter or replace functionalities of these building blocks.
This allows CORSIKA 8 to be used not only as a simulation program for air showers but also as a framework for other simulation purposes.


\subsection{The EGS4 computer code system}
\label{sec:egs4}

EGS4 (\textbf{E}lectron \textbf{G}amma \textbf{S}hower) is a standalone software package providing full Monte Carlo simulations of electromagnetic showers.
The first version of EGS, developed at the Stanford Linear Acceleration Center, has been published in 1978 with EGS4 being released in 1985.
EGS4 is capable of simulating the transport of electrons, positrons and photons in arbitrary geometries to obtain information on all secondary particles of an electromagnetic shower.
Particles with energies beginning at the $\si{\tera\electronvolt}$ range down to energies of only several $\si{\kilo\electronvolt}$ can be treated by EGS.
A detailed description of EGS4 is given by \cite{egs4}, a summary of the most important features of EGS4 and the changes made for its usage by CORSIKA is given as follows.

EGS4 is able to simulate the transport of particles in arbitrary media.
This is made possible by the data preparation code PEGS4 which, based on cross section tables for atomic numbers between 1 and 100, creates the data necessary for the propagation in the requested medium which are then used by EGS4.
After the tables for a specific medium have been created, EGS4 can be initialized either by receiving a monoenergetic particle to initiate an electromagnetic shower or by sampling an initial particle from a pre-defined distribution function.
During the propagation of electrons and positrons, the original version of EGS4 takes into account bremsstrahlung effects, collision losses, M{\o}ller or Bhabha scattering, multiple scattering as well as annihilation (for positrons) while for photons, electron-positron pair production, Compton scattering as well as photonuclear interactions are considered.

For photon propagation, all interactions are sampled individually with the photon traversing in a straight line with constant energy between two separate interactions.
When propagating charged leptons, EGS4 distinguishes between continuous and discrete interactions similar to the particle propagation in PROPOSAL as described in section \ref{sec:energy_loss_calculation}.
For every propagated particle, all produced secondary particles are stored and propagated as well, thus creating a full electromagnetic shower.

\subsection{Usage of EGS4 in CORSIKA}

Up to version seven, CORSIKA used EGS4 as an option to simulate the electromagnetic components of an atmospheric air shower.
To conform with additional requirements imposed by CORSIKA, several changes were made to the original EGS4 code which are described in detail in \cite{corsika_physics}.

Due to their importance for the muonic shower components, muon pair production by photons as well as the interaction of photons with protons and neutrons in the atmosphere were added.
To be able to describe interactions up to highest energies correctly, existing cross section data have been extended up to $\SI{e20}{\electronvolt}$.
Furthermore, the LPM effect, affecting bremsstrahlung and pair production at highest energies which is relevant for showers induced by high energy photons, has been integrated as well.

Since CORSIKA describes showers in a medium with an exponential density profile, the barometric density dependence of air influencing mean free path lengths as well as the density correction of ionization losses are relevant and have therefore been implemented in EGS4.

While EGS4 is natively able to treat the deflection of charged leptons in magnetic fields, approximations only valid for small deflection angles are used.
To comply with this requirement, the propagation step size is limited in cases where the deflection due to the earth's magnetic field would be too large.

To optimize computing times, the maximum step size between two interactions of electrons or positrons is increased whenever possible.
Furthermore, thinning measures are taken by removing particles from propagation if their probability to produce relevant secondary particles falls below a given threshold, preventing the calculation of irrelevant low-energy subshowers.