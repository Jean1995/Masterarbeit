\chapter{Introduction}

The field of astroparticle physics deals with the study of highly energetic charged particles, gamma-rays or neutrinos of extraterrestrial origin.
Information obtained from analyzing these messenger particles can provide insights into astrophysical sources, answer cosmological questions or improve the understanding of elementary particle physics.
One important experimental facility is the IceCube Neutrino Observatory, a large scale particle detector located at the geographical South Pole with to objective to search for high energy neutrinos.

To be able to analyze obtained experimental data, softwares providing adequate Monte Carlo simulations, i.e.\ simulations with both sufficient quantity and quality, are required. 
The lepton propagator PROPOSAL, the main focus of this work, is a Monte Carlo simulation software describing all interactions of highly energetic leptons in media.

In this work, extensions regarding two different aspects of PROPOSAL are presented.
The first extension deals with the description of rare processes, namely the production of muon pairs and the conversion of a charged lepton to a neutrino under exchange of a W boson.
Both processes are, although highly suppressed and therefore negligible for the average energy loss of muons, producing significant detector signatures.
To provide a framework for analyses regarding these signatures, both interactions are implemented as optional, additional processes in PROPOSAL.