\chapter{Propagation of electromagnetic showers}

\section{Electron and positron propagation in PROPOSAL}

\subsection{Ionization}

Ionization describes the inelastic collision of a particle with atomic electrons, leading to an energy loss of the primary particle.
For heavy, charged particles the average energy loss due to ionization is given by the Bethe formula.
PROPOSAL uses a modified Bethe formula, taking into account density correction effects, to describe the ionization losses for muons and electrons \cite{Kohne:2013zbq}.
However, the Bethe formula can't be applied for electrons and positrons due to their lower mass as well as the indistinguishability of the incoming electrons with atomic electrons.
Therefore, a separate treatment of the ionization losses for electrons and positrons is necessary.

For energy transfers $\nu$ significantly greater than the atomic excitation levels, the atomic electrons can be considered as free and in rest.
In this case, the ionization process is essentially electron-electron scattering ($e^- + e^- \rightarrow e^- + e^-$), called M{\o}ller scattering, or positron-electron scattering ($e^+ + e^- \rightarrow e^+ + e^-$), called Bhabha scattering.
The two feynman diagrams contributing in leading order for M{\o}ller scattering are shown in figure \ref{fig:feynman_moller}, the differential cross section \cite{PhysRev.93.38} is given by
%
\begin{equation}
	\begin{split}
	\label{eqn:moller}
	\left(\frac{\symup{d}\sigma}{\symup{d}v}\right)_{\!\!-} &= \frac{2 \pi r_e^2 Z \gamma}{\beta^2(\gamma - 1)^2} \biggl[ \frac{(\gamma - 1)^2}{\gamma^2} + \frac{1}{\epsilon} \left( \frac{1}{\epsilon} - \frac{2\gamma - 1}{\gamma^2} \right) \\ &+ \frac{1}{1 - \epsilon} \left( \frac{1}{1 - \epsilon} - \frac{2 \gamma - 1}{\gamma^2} \right) \biggr]
	\end{split}
\end{equation}
%
with
%
\begin{align}
	\epsilon &= \frac{v E}{E - m_e},& \gamma &= \frac{E}{m_e}, & \beta &= \sqrt{1 - \frac{1}{\gamma^2}}.
\end{align}

\begin{figure}
    \centering
    \begin{tikzpicture}
  \centering
   % Sizes
   \pgfmathsetmacro{\len}{0.05cm}
   \pgfmathsetmacro{\halflen}{\len/4}
   \pgfmathsetmacro{\vertexsize}{\len/20}
   \begin{feynman}
       % vertices
       \vertex (a) at (-1*\len, 0.5*\len);
       \vertex (b) at (0, 0);
       \vertex (c) at (1*\len, 0.5*\len);
       \vertex (d) at (0, -1.5*\len);
       \vertex (e) at (-1*\len, -2*\len);
       \vertex (f) at (1*\len, -2*\len);
  
       \vertex [label=\(\gamma\)] (label) at (-0.3 , -0.9 * \len);

       % draw diagram
       \diagram* {
         (a) -- [fermion] (b) -- [fermion] (c),
         (b) -- [photon] (d),
         (e) -- [fermion] (d) -- [fermion] (f)
       };  
       % labels
       \node[left] at (a) {$e^-$};
       \node[right] at (c) {$e^-$};
       \node[left] at (e) {$e^-$};
       \node[right] at (f) {$e^-$};

       % vertices
       \vertex (a2) at (6-1*\len, 0.5*\len);
       \vertex (b2) at (6+0, 0);
       \vertex (f2) at (6+1*\len, 0.5*\len);
       \vertex (d2) at (6+0, -1.5*\len);
       \vertex (e2) at (6-1*\len, -2*\len);
       \vertex (c2) at (6+1*\len, -2*\len);

       \vertex [label=\(\gamma\)] (label2) at (6-0.3 , -0.9 * \len);
  
       % draw diagram
       \diagram* {
         (a2) -- [fermion] (b2) -- [fermion] (c2),
         (b2) -- [photon] (d2),
         (e2) -- [fermion] (d2) -- [fermion] (f2)
       };  
       % labels
       \node[left] at (a2) {$e^-$};
       \node[right] at (c2) {$e^-$};
       \node[left] at (e2) {$e^-$};
       \node[right] at (f2) {$e^-$};


  \end{feynman}
\end{tikzpicture}
    \caption{Feynman diagram.}
    \label{fig:feynman_moller}
\end{figure}

For Bhabha scattering, the two feynman diagrams contributing in leading order are shown in figure \ref{fig:feynman_bhabha} and the differential cross section \cite{PhysRev.93.38} is given by
%
\begin{equation}
	\label{eqn:moller}
	\left(\frac{\symup{d}\sigma}{\symup{d}v}\right)_{\!\!+} = \frac{2 \pi r_e^2 Z \gamma}{(\gamma - 1)^2} \left[ \frac{1}{\beta^2 \epsilon^2} - \frac{B_1}{\epsilon} + B_2 - B_3 \epsilon + B_4 \epsilon^2 \right]
\end{equation}
%
with
%
\begin{align*}
	B_1 &= 2 - y^2, & B_2 &= (1 - 2y)(3 + y^2), \\ 
	B_3 &= (1-2y)^2 + (1 - 2y)^3, & B_4 &= (1 - 2y)^3, \\
	y &= (\gamma + 1)^{-1}.
\end{align*}

\begin{figure}
    \centering
    \begin{tikzpicture}
  \centering
   % Sizes
   \pgfmathsetmacro{\len}{0.05cm}
   \pgfmathsetmacro{\halflen}{\len/4}
   \pgfmathsetmacro{\vertexsize}{\len/20}
   \begin{feynman}
       % vertices
       \vertex (a) at (-1*\len, 0.5*\len);
       \vertex (b) at (0, 0);
       \vertex (c) at (1*\len, 0.5*\len);
       \vertex (d) at (0, -1.5*\len);
       \vertex (e) at (-1*\len, -2*\len);
       \vertex (f) at (1*\len, -2*\len);
  
       \vertex [label=\(\gamma\)] (label) at (-0.3 , -0.9 * \len);

       % draw diagram
       \diagram* {
         (a) -- [anti fermion] (b) -- [anti fermion] (c),
         (b) -- [photon] (d),
         (e) -- [fermion] (d) -- [fermion] (f)
       };  
       % labels
       \node[left] at (a) {$e^+$};
       \node[right] at (c) {$e^+$};
       \node[left] at (e) {$e^-$};
       \node[right] at (f) {$e^-$};

       % vertices
       \vertex (a2) at (6-1*\len, 0.5*\len);
       \vertex (b2) at (6-0.5, -0.75*\len);
       \vertex (f2) at (6+1*\len, 0.5*\len);
       \vertex (d2) at (6+0.5, -0.75*\len);
       \vertex (e2) at (6-1*\len, -2*\len);
       \vertex (c2) at (6+1*\len, -2*\len);
  
       % draw diagram
       \diagram* {
         (a2) -- [anti fermion] (b2) -- [anti fermion] (e2),
         (b2) -- [photon, edge label=\(\gamma\)] (d2),
         (c2) -- [anti fermion] (d2) -- [anti fermion] (f2)
       };  
       % labels
       \node[left] at (a2) {$e^+$};
       \node[right] at (c2) {$e^-$};
       \node[left] at (e2) {$e^-$};
       \node[right] at (f2) {$e^+$};


  \end{feynman}
\end{tikzpicture}
    \caption{Feynman diagram.}
    \label{fig:feynman_bhabha}
\end{figure}

For energy transfers $\nu$ in the same order of magnitude as the atomic excitation levels, the explicit excitation probabilities $p(E, v)$ need to be considered.
With a threshold transfer energy $E_{\text{thres}}$ that is above the atomic excitation levels but below the energy cut in PROPOSAL, the continuous energy loss per distance due to ionization can be written as
%
\begin{equation}
	\label{eqn:ioniz_sum}
	-\frac{\symup{d}E}{\symup{d}X} \propto \int_{0}^{\sfrac{E_{\text{thres}}}{E}} v \cdot p(E, v) \, \symup{d}v + \int_{\sfrac{E_{\text{thres}}}{E}}^{v_{\text{cut}}} v \left(\frac{\symup{d}\sigma}{\symup{d}v}\right)_{\!\!\pm} \, \symup{d}v. 
\end{equation}
%
Assuming an appropriate value for $E_{\text{thres}}$ and using suitable approximations, \eqref{eqn:ioniz_sum} yields
%
\begin{equation}
	- \frac{\symup{d}E}{\symup{d}X} = \frac{2 \pi r_e^2 m_e n}{\beta^2} \left[ \log{\left( \frac{2 m_e (\tau + 2)}{I} \right) + F^{\pm}(\tau, \Delta) - \delta } \right]
\end{equation}
%
also known as the Berger-Seltzer formula \cite{Hirayama:2005zm}.
\todo{Prefactors of the equations. Captions of plots and feynman diagrams. Kinematic limits for the cross sections.}
\begin{figure}
    \centering
    \includegraphics[scale=1]{build/dEdx_ionization.pdf}
    \caption{Plot.}
    \label{fig:dEdx_ionization}
\end{figure}

\subsection{Bremsstrahlung}

\subsection{Electron-positron annihilation}

\section{Photon propagation in PROPOSAL}

\subsection{Pair production by photons}

\subsection{Compton scattering}

\subsection{Cross checks}

\subsection{Electromagnetic shower propagation with PROPOSAL}