\chapter{Atomic form factors for photo pair production}
\label{sec:atomic_form_factors}

Atomic form factors are necessary to describe the effects on electron-positron pair production by photons due to the field of atomic electrons.
These effects are firstly the screening of the nuclear Coulomb field, described by the elastic atomic form factor $G_2^{\text{el}}$, and secondly events where the photon scatters from the electron field screened by the nucleus, described by the inelastic atomic form factor $G_2^{\text{inel}}$.
The contributions from the atomic form factors are only relevant for electron-positron pair production at small production angles which is the usual use case for high energy photon propagation.
To obtain the functions $\varphi_i$, $\psi_i$ and $X = X_{\text{el}} + X_{\text{inel}}$, which are used to parametrize the information from these atomic form factors, integrations of $G_2$ over kinematic variables are necessary.
The different expressions used for these functions, varying with $Z$, are presented in the following.

\section{Hydrogen and Helium}

For $Z=1$, the atomic form factors are known exactly while for $Z=2$, an approximative model leading to a similar analytic expression can be used.
These form factors can be integrated analytically leading to the expressions \cite{RevModPhys.46.815}
%
\begingroup
\allowdisplaybreaks
\begin{align}
	\begin{split}
		\varphi_1 &= \frac{4}{3} \ln(Z) + 4 \ln\left(\frac{1}{2 \eta \alpha}\right) + \frac{13}{3} - 2 \ln(1 + C^2) \\&\quad- \frac{13}{2} C \arctan(C^{-1}) + \frac{1}{6}\frac{1}{1 - C^{-2}},
	\end{split}
	\\[2ex]
	\begin{split}
		\varphi_2 &= \frac{4}{3} \ln(Z) + 4 \ln\left(\frac{1}{2 \eta \alpha}\right) + \frac{11}{3} - 2 \ln(1 + C^2) \\&\quad+ 25 C^2 (1 - C \arctan(C^{-1})) - 14 C^2 \ln(1 + C^{-2}),
	\end{split}
	\\[2ex]
	\begin{split}
		\psi_1 &= \frac{8}{3} \ln(Z) + 4 \ln\left(\frac{1}{2 \eta \alpha}\right) + \frac{23}{3} - 2 \ln (1 + C^2) \\&\quad- 17.5 C \arctan{C^{-1}} + 8 C^2 \ln(1 + C^{-2}) - \frac{1}{6}\frac{1}{1 + C^{-2}},
	\end{split}
	\\[2ex]
	\begin{split}
		\psi_2 &= \frac{8}{3} \ln(Z) + 4 \ln\left(\frac{1}{2 \eta \alpha}\right) + \frac{21}{3} - 2 \ln (1 + C^2) \\&\quad- 105 C^2 (1 - C \arctan(C^{-1})) + 50 C^2 \ln(1 + C^{-2}) \\&\quad- 24 C^2 \biggl[ - \ln(C^2) \ln(1 + C^{-2}) + \Phi(1 + C^{-2}) - \Phi(1) \biggr],
	\end{split}
	\\[2ex]
	\begin{split}
		X_{\text{el}} &= Z^2 \biggl[ 2 \ln\left(\frac{m_e}{\delta}\right) - \ln(1 + B^2) + \frac{1}{6} - \frac{4}{3} \frac{1}{1 + B^2} \\&\quad+ \frac{1}{6} \frac{1}{(1 + B^2)^2} \biggr], 
	\end{split}
	\\[2ex]
	\begin{split}
		X_{\text{inel}} &= Z \biggl[ 2 \ln\left(\frac{m_e}{\delta}\right) - \ln(1 + B^2) + \frac{11}{6} - 4 B^{-2} \ln(1 + B^2) \\&\quad+ \frac{4}{3} \frac{1}{1 + B^2} - \frac{1}{6} \frac{1}{(1 + B^2)^2} \biggr],
	\end{split}
\end{align}
\endgroup
%
with
%
\begin{align*}
	\delta &= \frac{m_e^2}{2 E_{\gamma} x (1-x)}, & C &= \frac{\delta}{2 \alpha m_e \eta}, & x &= \frac{E_{-}}{E_{\gamma}}, \\ t_{\text{min}}^{\prime} &= \left( \frac{m_e^2 (1+l)}{2 E_{\gamma} x (1-x)} \right)^2, & B &= \frac{2 \alpha m_e \eta}{\sqrt{t_{\text{min}}^{\prime}}}, & l &= \frac{E_-^2 \theta^2}{m_e^2},
\end{align*}
%
where $E_{\gamma}$ is the energy of the photon, $E_-$ the energy of the produced electron, $\theta$ the angle between the photon and the produced electron and $\eta$ is defined as
%
\begin{align*}
	\eta &= 
	\begin{cases}
		1 & \text{if $Z = 1$},\\
		1.6875 & \text{if $Z = 2$},\\
	\end{cases}
\end{align*}
%
with the dilogarithm function $\Phi(x)$.

\section{Lithium and Beryllium}

For $Z=3$ and $Z=4$, the atomic form factors $G_2$ are only known numerically.
Instead of referring to these values, a simpler approximation for the form factors is used.
The parameters of this approximation are chosen in such a way that the results for $G_2$ are exact in the limits of complete screening and no screening.
Based on that, the functions $\varphi_i$, $\psi_i$ and $X = X_{\text{el}} + X_{\text{inel}}$ are \cite{RevModPhys.46.815}
%
\begingroup
\allowdisplaybreaks
\begin{align}
	\label{eqn:phi_1}
	\begin{split}
		\varphi_1 &= 2\left(1 + \ln(a^2 Z^{\sfrac{2}{3}} m_e^2)\right) - 2 \ln(1 + b^2) - 4 b \arctan(b^{-1}),
	\end{split}
	\\[2ex]
	\begin{split}
		\varphi_2 &= 2 \left(\frac{2}{3} + \ln(a^2 Z^{\sfrac{2}{3}} m_e^2 )\right) - 2 \ln(1 + b^2) \\&\quad+ 8 b^2  \biggl[ 1 - b \arctan(b^{-1}) - 0.75 \ln(1 + b^{-2}) \biggr],
	\end{split}
	\\[2ex]
	\begin{split}
		\psi_1 &= 2\left(1 + \ln(a^{\prime2} Z^{\sfrac{4}{3}} m_e^2)\right) - 2 \ln(1 + b^{\prime2}) - 4 b^{\prime} \arctan(b^{-1}),
	\end{split}
	\\[2ex]
	\begin{split}
		\psi_2 &= 2 \left(\frac{2}{3} + \ln(a^{\prime2} Z^{\sfrac{4}{3}} m_e^2 )\right) - 2 \ln(1 + b^{\prime2}) \\&\quad+ 8 b^{\prime2}  \biggl[ 1 - b \arctan(b^{\prime-1}) - 0.75 \ln(1 + b^{\prime-2}) \biggr],
	\end{split}
	\\[2ex]
	\begin{split}
		X_{\text{el}} &= Z^2 \biggl[ \ln\left( \frac{a^2 m_e^2 (1 + l)^2}{a^2 t_{\text{min}}^{\prime} + 1}\right) - 1 \biggr],
	\end{split}
	\\[2ex]
	\label{eqn:X_inel}
	\begin{split}
		X_{\text{inel}} &= Z \biggl[ \ln\left( \frac{a^{\prime2} m_e^2 (1 + l)^2 }{a^{\prime2} t_{\text{min}}^{\prime} + 1 } \right) - 1\biggr],
	\end{split}
\end{align}
\endgroup
%
with
%
\begin{align*}
	a &= 
	\begin{cases}
		\frac{100}{m_e} Z^{-\sfrac{1}{3}} & \text{if $Z = 3$},\\
		\frac{106}{m_e} Z^{-\sfrac{1}{3}} & \text{if $Z = 4$},\\
	\end{cases}\\
	a^{\prime} &= 
	\begin{cases}
		\frac{418.6}{m_e} Z^{-\sfrac{2}{3}} & \text{if $Z = 3$},\\
		\frac{571.4}{m_e} Z^{-\sfrac{2}{3}} & \text{if $Z = 4$},\\
	\end{cases}
\end{align*}
and $b = a \delta$, $b^{\prime} = a^{\prime} \delta$.
%

\section{Heavier elements}

For $Z>4$, the Thomas-Fermi model of the atom is used to evaluate the atomic form factors.
The parameters in \eqref{eqn:phi_1} to \eqref{eqn:X_inel} were adapted to fit the numerical values obtained from the Thomas-Fermi model, yielding the expressions \cite{RevModPhys.46.815}
%
\begingroup
\allowdisplaybreaks
\begin{align}
	\begin{split}
		\varphi_1(\gamma) &= 20.863 - 2 \ln(1 + (0.55846 \gamma)^2) \\&\quad- 4 \biggl[ 1 - 0.6 \exp(-0.9\gamma) - 0.4 \exp(-1.5 \gamma) \biggr],
	\end{split}
	\\[2ex]
	\begin{split}
		\varphi_2(\gamma) &= \varphi_1(\gamma) - \frac{2}{3} \frac{1}{1 + 6.5\gamma + 6\gamma^2},
	\end{split}
	\\[2ex]
	\begin{split}
		\psi_1(\epsilon) &=  28.34 - 2\ln(1 + (3.621 \epsilon)^2) \\&\quad- 4 \biggl[ 1 - 0.7 \exp(- 8 \epsilon) - 0.3 \exp(-29.2\epsilon) \biggr],
	\end{split}
	\\[2ex]
	\begin{split}
		\psi_2(\epsilon) &= \psi_1(\epsilon) - \frac{2}{3} \frac{1}{1 + 40\epsilon + 400\epsilon^2},
	\end{split}
	\\[2ex]
	\begin{split}
		X_{\text{el}} &= Z^2 \biggl[ \ln\left( \frac{a^2 m_e^2 (1 + l)^2}{a^2 t_{\text{min}}^{\prime} + 1}\right) - 1 \biggr],
	\end{split}
	\\[2ex]
	\begin{split}
		X_{\text{inel}} &= Z \biggl[ \ln\left( \frac{a^{\prime2} m_e^2 (1 + l)^2 }{a^{\prime2} t_{\text{min}}^{\prime} + 1 } \right) - 1\biggr],
	\end{split}
\end{align}
\endgroup
%
with
%
\begin{align*}
	\gamma &= \frac{200 \delta}{m_e Z^{\sfrac{1}{3}}}, & \epsilon &= \frac{200 \delta}{m_e Z^{\sfrac{2}{3}}}, & a &= 111.7 \frac{Z^{-\sfrac{1}{3}}}{m_e}, & a^{\prime} &= 724.2 \frac{Z^{-\sfrac{2}{3}}}{m_e}.
\end{align*}
%

\chapter{Constants and variables}
\label{sec:constants}

The variables listed in table \ref{tab:constants} are used for this thesis as well as in PROPOSAL, the corresponding numerical values are obtained from \cite{PhysRevD.98.030001}.

\begin{table}
	\centering
	\caption[]{Defintions and numerical values of constants and variables.}
	\label{tab:constants}
	\sisetup{
  		table-format=1.9e+2,
  		exponent-product={\cdot}
  		}	
	\begin{tabular}{l l l}
		\toprule
		{Variable} & {Definition} & {Numerical value} \\	
		\midrule
		$\alpha$ & \text{Fine-structure constant} & \num{0.0072973525664} \\
		$N_A$ & \text{Avogadro Constant} & \SI{6.022140857e23}{\per\mol} \\
		$m_e$ & \text{Electron mass} & \SI{0.5109989461}{\mega\electronvolt} \\
		$m_{\mu}$ & \text{Muon mass} & \SI{105.6583745}{\mega\electronvolt} \\
		$r_e$ & \text{Classical electron radius} & \SI{2.8179403227e-13}{\centi\metre} \\
		$r_{\mu}$ & \text{Classical muon radius} & $r_e \cdot m_e \cdot m_{\mu}^{-1}$ \\
		$Z$ & \text{Atomic number} \\
		$A$ & \text{Atomic mass number} \\
		$e$ & \text{Euler's number} & \num{2.718281828459045} \\
		$\pi$ & & 3.141592653589793 \\
		\bottomrule
	\end{tabular}
\end{table}

\chapter{Reproducibility}

To create the figures presented in this thesis, the programming language \emph{Python 3.8.1}, together with the Python packages \emph{matplotlib v3.1.2} \cite{Hunter:2007}, \emph{NumPy v1.18.0} \cite{oliphant2006guide} as well as \emph{SciPy 1.4.1} \cite{2020SciPy-NMeth}, has been used.
Feynman diagrams have been produced using the TikZ-Feynman \LaTeX-package \cite{Ellis_2017}.
The PROPOSAL version used to produce all results has been tagged and is publicly available on GitHub under the link \url{https://github.com/Jean1995/PROPOSAL/releases/tag/thesis}.

All results from this thesis may be reproduced using the scripts available on GitHub under the link \url{https://github.com/Jean1995/Masterarbeit}, this requires installing the corresponding versions of the above-mentioned packages. 
If \emph{GNU Make} and a \LaTeX\ compiler are available, executing the console command \texttt{make} reproduces all plots as well as the thesis itself.
